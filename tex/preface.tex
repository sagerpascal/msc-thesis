\small
I begin this preface by providing readers with insights about the background of this Master's thesis.
It is important to understand this thesis as a first step towards a broader endeavour - as preparatory work for a potential dissertation.
Hence, this thesis should not be seen as self-contained work but rather as a (hopefully exciting) conceptual foundation that will be further developed and refined in the upcoming years.

Many aspects of the proposed framework still require clarification or validation through experimental evidence. I kindly ask you, the reader of this thesis, to understand if certain concepts have not yet been fully explored. I hope you can see the value of these ideas and that they will arouse your curiosity.

Upon successful assessment of this thesis and achieving a satisfactory grade, I am entitled to use the title ``Master of Science''.
Thus, I should ``master'' science or at least be able to work scientifically.
Science can be defined as the systematic analysis of the real or virtual world through observations and experiments and the further development of existing technology.
While this definition sounds straightforward, working successfully in science requires a lot of experience and commitment.
Throughout my Master's studies, I have had the privilege of delving deeper into science and applying the obtained knowledge in research projects at the Centre for Artificial Intelligence (CAI) of the Zurich University of Applied Sciences (ZHAW).
This daily engagement with scientific work has proven extremely valuable in writing this thesis and will undoubtedly help me in the future.

However, I still faced challenges in writing this thesis:
As someone with an industry and engineering background, I tend to approach tasks with a ``do-it'' mentality and intuitively refine my ideas with hands-on experimentation. 
In this Master's thesis, it took me some time to make the transition from this do-it mindset to that of a researcher who is also strong in theoretical foundations. Thanks to this Master's thesis, I believe I now balance implementation and studying theory much better.

Completing this thesis spanned an entire year, the maximum duration allowed by my university.
The process of formulating and refining the theory was punctuated by setbacks and experiments that did not produce the desired results.
After approximately seven months of work, I developed two biologically inspired models that showed promising performance. 
In fact, I had even written an entire thesis about these models, which, with some improvements, could have been submitted.
However, as these concepts did not entirely convince me as a foundation for a dissertation, I decided to discard my work and pursue a new approach only five months before the submission deadline.
Consequently, this thesis contains only a fraction of the conducted experiments but builds on the experience gained from earlier failed attempts.
Nevertheless, I am convinced that this decision was the right one and that the quality and consistency of a thesis should outweigh the quantity.
I hope you, dear reader, agree with this perspective and understand the constraints that led to incomplete experimentation.

Throughout my Master's studies and my work at the CAI, I  have had the privilege to be supported and mentored by Prof. Dr. Thilo Stadelmann, Head of the CAI.
He emphasises the idea that (also in accordance with his \href{https://stdm.github.io/Great-methodology-delivers-great-theses/}{blog-post}) ``Great methodology delivers great theses''.
While it is undoubtedly desirable to achieve an exceptional result in a thesis, it is equally, if not more, important to articulate the rationale behind the methodology, justify choices and demonstrate the limitations. 
Moreover, scientific breakthroughs always require courage to try something completely new, even if this means that the work may not lead to outstanding results worth publishing.
These principles have guided me in writing this thesis, and I hope that readers will be able to understand my thought process.

This thesis spans the fields of computer science and neuroscience and attempts to make clear connections between these areas so that readers from both disciplines can understand my arguments. 
However, this also means that some aspects are described rather extensively.
So if you as a reader consider yourself an expert in one of these fields, feel free to skip (parts of) the fundamentals in \chref{fundamentals}.

At this point, I also want to thank colleagues and friends for supporting this thesis.
Foremost, I think of my mentor Prof. Dr. Thilo Stadelmann, who got me excited about AI years ago and later introduced me to research.
He always encourages creative ideas, thinking outside the box, and striving for greatness.
Thank you for your support, help, and guidance; I have grown personally and professionally.
Further thanks go to Dr. Jan Deriu. 
He has always helped to translate abstract ideas into concrete algorithms and to get them running.
It's impressive how your understanding of deep learning can make complex problems look so simple.
To Prof. Dr. Christoph von der Malsburg for his seemingly endless patience in introducing me to neuroscience.
You have inspired me regularly with ideas and opened up a new way of thinking about deep learning (this was also the inspiration for Geoffrey Hinton's quote at the beginning of this thesis, although I wouldn't presume to say that this thesis will change the future).
Even though we couldn't implement all of your ideas in this thesis, I learned a lot in our discussions and hope that I will be able to tackle more of your thoughts in the future.

The most profound thanks, however, goes to my family, who made this journey possible for me:
To my parents, who supported and encouraged me in every way.
To my younger brother, who inspired me to study.
To my wife and son, who have been understanding and supportive and have always been the perfect counterbalance to the daily routine.
Without the support of my family, I would never have been able to embark on this academic path.
\normalsize

