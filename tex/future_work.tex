%% future_work.tex

\section{Useful Representations}\seclbl{useful_representations}
In this thesis I describe an approach to generate representations of a visual scene.
However, the question how useful this representations are still remains\sidenote{despite for the usual ML tasks such as image classifications}.
The ultimate goal of perception systems is to build a simple and informative model of the external world.
This internal model should not be too detailed but include all behavior-relevant information.
Thus, the internal model should allow an agent to take the optimal action to fulfil a given task.

It is known from the study of animals that both eye movements and the behavioral state influence the responses of neurons in the visual cortex \sidecite{Keller_Bonhoeffer_Hübener_2012}.
Thus, animals integrate their action (i.e. the movement they are doing) with currently incoming sensory signals to predict future sensory inputs.
The internal copy of an outflowing movement-producing signal generated by an organism's motor system is also known as efference copy.
Keurti et al. \sidecite{Keurti_Pan_Besserve_Grewe_Schölkopf_2022} argue that such efference copies are useful to learn \emph{useful} latent representations perceived by the visual system.
They translated this idea into an AI-based system by allowing an agent to interact with the environment and to observe its state to build internal representations.

If useful latent representations of a visual scene are considered to be a world-model with all behavior-relevant information, it should be evaluated whether the representations obtain by the proposed model correspond to this definition.
If not, which is likely according to the author's intuition, the latent representations should be optimized accordingly.
This implies that the existing perception system should be extended by cognition.
Thus, the AI system needs an embodiment to interact with the environment.
For example, this can be simulated with a reinforcement learning based agent.
The training process could be explicitly modeled by predicting future states based on a given state and possible actions before the action is executed and the actual outcome in the world model is observed.
This procedure corresponds to the perception-action episode that was proposed by LeCun \sidecite{LeCun_AMI}.
He divides the process in seven steps;
(i) First, the perception system extracts a representation of the current state of the world \(s[0]=P(x)\). (ii) The actor then proposes an initial sequence of actions \((a[0], ..., a[t], ... a[T])\) that is evaluated by the world model. (iii) The world model in turn predicts likely sequence of world state representations resulting from the proposed action sequence \((s[1], ..., s[t], ... s[T])\). (iv) A cost model estimates the total costs for each state sequence as a sum over time steps \(F(x)=\sum_{t=1}^{T}C(s[t])\). (v) Based on the cost predictions, the actor proposes the action sequence with the lowest costs. (vi) The actor then executes one or a few actions (and not the entire action sequence) and the entire process is repeated. (vii) Additionally, every action, the states and associated costs are stored in a short-term memory that can be used to optimize the system.

TODO: Write more about Paper from Grewe \sidecite{Keurti_Pan_Besserve_Grewe_Schölkopf_2022}


TODO: Stand jetzt sind es eher neue Module wie lateral connections -> gute Repräsentationen benötigen aber neue Architekturen, z.B. eine Hierarchie (Tiere -> Säugetiere, Reptilien, -> ...) und eine Menge an Actions, die jedes Element in der Hierarchie ausführen kann (z.B. drehen kann jedes Tier, Gehen nur die Landtiere und Schreiben nur der Mensch) -> Schreibe konkrete Vorschläge für Folgearbeiten!



