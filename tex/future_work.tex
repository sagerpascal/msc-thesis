
\section{Discussion}



\section{Future Work}

% S1: Negative Hebbian Learning um Pattern zu vergessen, Alternative Cells


\subsection{Multi-Modality}\seclbl{framework_multi_modality}
This work focuses on a framework for computer vision. However, the architecture has broader applicability and can be used for processing different sensor signals and be used in multimodal settings \cite{ngiam_multimodal_2011, liu_learn_2018, baltrusaitis_multimodal_2019}.
Having similar cell architectures processing different kind of signals is also in line with findings from neuroscience \sidecite{mountcastle_organizing_1978, mountcastle_columnar_1997}.

In the case of images, net fragments in \emph{S1} represent learned visual patterns that are part of an object's surface and are mapped with protection fibres to object prototypes that describe the visual appearance of objects. 
The same architectural structure can be applied to other types of signals. For example, an alternative sensory system could perceive audio signals. In this scenario, the local support in \emph{S1} would extend over nearby frequency ranges and time intervals. Consequently, phonemes or syllables could be learned locally and represented by net fragments. In the second stage (\emph{S2}), a sequence of phonemes or syllables could be mapped onto word prototypes.

Different sensory systems could even have separate domain-specific \emph{S1} stages in a multimodal setting, while the prototypes in \emph{S2} could be shared across modalities. This arrangement would allow to integrate different sensor signals and facilitates the creation of internal object representations with multiple modalities.



\subsection{Alternative Cells in \emph{S1}}
Hemmung zwischen Output Channels als Competition

Competition: Welcher Channel passt am besten zu den Daten


Schwer umzusetzen: Zu Beginn alle identisch, Symmetry muss gebrochen werden (Gesetzt der grossen Zahlen -> nur Durchschnitt reicht nicht)


Competition nur nötig, wenn mehrere Channels aktiv sind


\subsection{Negative Hebbian Learning in \emph{S1}}


Was wenn falsche Daten / Statistiken gelernt werden -> Vergessen mit Negative Hebbian Learning

Auch für Alternative Zellen hilfreich (z.B. Zellspaltung, dann spezialisiert sich jede Zelle auf etwas)

 Challenge: Wenn an gewissen Stellen negative Correlation dominiert


\subsection{Leverage Multiple Views}


