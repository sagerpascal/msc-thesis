%% related_work.tex

\section{Natural Intelligence}
This thesis is inspired by the work ''A Theory of Natural Intelligence` from von der Malsburg et al. \sidecite{von_der_Malsburg_Stadelmann_Grewe_2022}.
Therefore, we dedicate this section to summarize their work in detail.

According \cite{von_der_Malsburg_Stadelmann_Grewe_2022}, the process of learning is influenced by ``nature'', ``nurture'', and ``emergence''\sidenote{nature refers to the influence of genes and evolution, nurture to the influence of experience and education}.
They point out that human genome (as of nature) only contain 1GB of information \sidecite{hbcrd} and humans only absorb a few GB into permanent memory over a lifetime (as of nurture) but it requires about 1PB to describe the connectivity in human brain.
Therefore, it is important to distinguish the amount of information to describe a structure from the amount of information needed to generate it.
Similar, nature and nurture only require a few GB to construct, respectively instruct the entire human brain.
Therefore, they argue that the human brain must be highly structured (i.e. nature and nurture ``generate'' the human brain by selecting from a set pre-structured patterns).
The authors call the process of generating the highly structured network in the human brain the ``Kolmogorov \sidecite{Kolmogorov_1998} Algorithm of the Brain''\sidenote{as the Kolmogorov complexity describes the number of bits required by the shortest algorithm that can generate the structure}.
Network self-organization is the only mechanism that has not yet been disproved by experiments as the brains Kolmogorov algorithm \sidecite{Willshaw_VonDerMalsburg_1976, Willshaw_VonDerMalsburg_1979}.
This mechanism loops between activity and connectivity, with activity acting back on connectivity through synaptic plasticity until a steady state, called an attractor network, is reached.
The consistency property of an attractor network means that a network has many alternative signal pathways between pairs of neurons \sidecite{Malsburg_1987}.
Thus, the brain develops as an overlay of attractor networks called net-fragments \sidecite{vonderMalsburg_2018}.
Net-fragments consist of small sets of neurons, whereby each neuron can be part of several net fragments.
The network self-organization has to start from an already established coarse global structure which is improved in a coarse-to-fine manner to avoid being caught in a local optima.

Also, von der Malsburg et al. \cite{von_der_Malsburg_Stadelmann_Grewe_2022} discuss scene representation (i.e. how a scene is represented in the brain) even tough they point out that this is a contested concept \sidecite{freeman1990representations}.
Scene representation is a organization framework to put abstract interpretation of scene layouts, elements, potential actions, and emotional responses in relation.
The details are not rendered as in photographic images but the framework supports the detailed reconstructions of narrow sectors of the scene.
The basic goal if learning is to integrate a behavioral schema into the flow of scene representations.
They propose the hypothesis that the network structure resulting from self-organization together with the neural activation in the framework of scene representation are the inductive bias that tunes the brain to the natural environment.

Finally, they discuss how net fragments can be used to implement such structures and processes using vision as an example.
They point out that a neuron is grouped in one or multiple net fragments through network self-organization.
The net fragments can be considered as filters that detect previously seen patterns in the visual input signal.
An object is represented by multiple net fragments, where each fragment responds to the surface of that object and has shared neurons and connections with other net fragments representing that object.
Thus, net fragments render the topological structure of the surfaces that dominate the environment.
Von der Malsburg et al. \cite{von_der_Malsburg_Stadelmann_Grewe_2022} propose that net fragments represent shape primitives which can adapt to the shape of actual objects\sidenote{adapt in spite of metric deformations, depth rotation, and position}.
Shifter circuits are one possible implementation of networks that enable invariant responses to the position- and shape-variant representations \sidecite{Arathorn_2002, Olshausen1995}.
They are composed of net-fragments that can be formed by network self-organization \sidecite{Fernandes_vonderMalsburg_2015}.
Ref. \cite{von_der_Malsburg_Stadelmann_Grewe_2022} also argue that net fragments are the compositional data structure used by the brain.
A hierarchy of features may be represented by nested net fragments of different size.
Complex objects, such as mental constructs, can thus be seen as larger net fragments composed as mergers of pre-existing smaller net fragments.


\section{Self-Organization}
The human brain is self-organizing \sidecite{kelso1995dynamic}.
Self-organization is the process by which systems consisting of many units spontaneously acquire their structure or function without interference from a external agent or system.
The absence of a central control unit allow self-organizing systems to quickly adjust to new environmental conditions.
Additionally, such systems have in-built redundancy with a high degree of robustness as they  are made of many simpler individual units.
These individual units can even fail without the overall system breaking down.
Dresp \sidecite{Dresp2020SevenPO} describes seven clearly identified properties of self-organization in the human brain: (i) modular connectivity, (ii) unsupervised learning, (iii) adaptive ability, (iv) functional resiliency, (v) functional plasticity, (vi) from-local-to-global functional organization, and (vii) dynamic system growth.

Before summarizing the literature specific to self-organization of neural networks, the general literature on self-organization with a focus on deep learning is described in the following.
Many of these fundamental algorithms for self-organization serve as inspiration for how ANNs can be designed to be self-organizing.

In nature, groups of millions units that solve complex tasks by using only local interactions can be observed.
For example, ants can navigate difficult terrain with a local pheromone-based communication and thus form a collective type of intelligence.
Such observations inspired researchers to build algorithms which are based on local communication and self-organization, for example ant colony optimization algorithms \sidecite{dorigo1997ant}.
DeepSwarm \sidecite{Byla_Pang_2020} is a neural architecture search method that uses this algorithm to search for the best neural architecture.
This methods achieves competitive performance on rather small datasets such as MNIST, Fashion-MNIST, and CIFAR-10.

%Robotic is another research area that uses ideas from collective intelligence such as self-assembly or self-organization.
%For example, swarm systems consist of multiple robots that work together to solve complex tasks \sidecite{Hamann_2018}.
%A famous example of self-assembling robots was presented in 2014 by Rubenstein et al. \sidecite{Rubenstein_Cornejo_Nagpal_2014}.
%They teach kilobots\sidenote{kilobots are 3.3cm tall low-cost swarm robots developed at Harvard University} to self-assemble into target shapes such as letters or stars solely based on local communication between robots.
%However, the kilobots still rely on hand-crafted algorithms to determine their position in the global coordinate system.

%TODO: Write more about swarm intelligence or delete thie paragraph above (does not really fit in here....)

Cellular Automata mimic developmental processes in multi-cell organisms.
They contain a grid of similar cells with an internal state which is updated periodically.
The transition from a given state to a subsequent state is defined by some update rules.
Such automata can be used to study biological pattern formations \sidecite{Wolfram1984} or physical systems \sidecite{VICHNIAC198496}.
Neural Cellular Automata \sidecite{Wulff1992LearningCA} use neural networks to learn these update rules.


%https://arxiv.org/abs/2006.06902
%https://proceedings.neurips.cc/paper/2019/file/1e6e0a04d20f50967c64dac2d639a577-Paper.pdf


%https://www.nature.com/articles/srep12866
%https://sebastianrisi.com/self_assembling_ai/ + Video -> very good points at beginning + self organization (start at min. 30 also about self organizing networks)

% Related work: Hedge Backpropagation: https://arxiv.org/abs/1711.03705


% Summarize work from Claude -> only if relevant...










