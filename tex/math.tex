\section{Notation}

Depending on the context, a variable can be a scalar value, a vector, or a matrix. The following formatting is used:

\renewcommand{\arraystretch}{1.3}
\begin{tabular}{ p{5cm} p{2cm} p{7cm} }
	\textbf{Formatting} & \textbf{Example} & \textbf{Meaning}\\
	\hline
  	No formatting, lower case & $a$ & A scalar value.\\
  	Bold, lower case & $\boldsymbol{a}$ & A vector.\\
  	Bold, upper case & $\boldsymbol{A}$ & A matrix.\\
  	Curved brackets & $a(\cdot)$ & A function, where $(\cdot)$ is a placeholder for a variable.\\
  	Rectangular brackets & $a[t]$ & Variable $a$ at time $t$.\\
   Superscript rectangular brackets & $a^{[l]}$ & $l$ is the index of the layer, e.g. $\boldsymbol{W}^{[l]}$ is the weight matrix of layer $l$.\\
\end{tabular}

\section{Variables}

\begin{tabular}{ p{3cm} p{11cm} }
	\textbf{Variable} & \textbf{Meaning}\\
	\hline
	$\eta$ & The learning rate of an optimisation algorithm.\\
    $\kappa$ & The number of alternative cells.\\
    $\psi$ & Estimated pre- or post-synaptic activity of a cell, required for Hebbian learning (c.f. \eqref{hebb_2}).\\
    $\rho$ & Upper limit for cell support before the support is decreased.\\
	$\boldsymbol{a} = (a_1, ..., a_k)$ & The output of an intermediate layer of a multi-layer network with $k$ neurons, the output of the last layer is typically denoted as $\boldsymbol{\hat{y}}$. $a_i$ is the output of an activation function of a single neuron.\\
	$\boldsymbol{b} = (b_1, ..., b_k)$ & The bias of a network layer with 
    $k$ neurons. $b_i$ is the bias of a single neuron.\\
    $C$ & The capacity of a Hopfield network, i.e. the number of patterns that can be stored or the number if channels from an input matrix.\\
    $h$ & Hidden state of the memory, storing object prototypes.\\
    $H$ & The height of an image.\\
    $J$ & The Jaccard similarity between two binary vectors (c.f. \eqref{jaccard}).\\
    $k$ & The number of neurons within a layer or kernel.\\
    $L$ & The number of layers of a network.\\
    $n$ & The length (size) of a vector, for example, an input of length $n$ is defined as $\boldsymbol{x} = (x_1, ..., x_n)$. $n_l$ is also used for the support distance of lateral connections.\\
    $o$ & A single neuronal cell.\\
    $P$ & The probability of observing a variable.\\
    $t$ & The current timestep.\\
    $T$ & The total number of timesteps.\\
    $w_{ij}$ & The weight between neuron $i$ and neuron $j$.\\
	$\boldsymbol{w} = (w_1, ..., w_n)$ & A weight vector of a neuron, mapping an input of length $n$ to a scalar value.\\
    $W$ & The width of an image.\\
	$\boldsymbol{W} = (\boldsymbol{w}_1, ..., \boldsymbol{w}_k)$ & A weight matrix of a network layer.\\
    $\boldsymbol{x} = (x_1, ..., x_n)$ & An input sample that is fed into a model or a network layer, typically a vector of length $n$.\\
    $Z$ & Capacity of the memory.\\
 \end{tabular}
 
 \begin{tabular}{ p{3cm} p{11cm} }
	\textbf{Variable} & \textbf{Meaning}\\
	\hline
     $\boldsymbol{y}$ & The expected output of a model or a network layer (i.e. the ground truth), typically a vector ($\boldsymbol{y}$) for a multi-class classification task or a scalar value ($y$) for a regression or single-class classification task.\\
 	$\boldsymbol{\hat{y}}$ & The actual output of a model or a network layer (i.e. the prediction), typically a vector ($\boldsymbol{\hat{y}}$) for a multi-class classification task or a scalar value ($\hat{y}$) for a regression or single-class classification task.\\
	$\boldsymbol{z} = (z_1, ..., z_k)$ & The output of the aggregation functions $g_1(\cdot), ..., g_n(\cdot)$ of a network layer of length $k$. $z_i$  is the output of the aggregation function of a single neuron.\\
\end{tabular}


\section{Functions}

\begin{tabular}{ p{3cm} p{11cm} }
	\textbf{Function} & \textbf{Meaning}\\
	\hline
	$f(\cdot)$ & An activation function such as $\sigma(\cdot)$, $(\cdot)^{+}$, or $\tanh(\cdot)$ (c.f. \eqref{act_functions}).\\
    $F(\cdot)$ & The free energy function.\\
	$g(\cdot)$ & The aggregation function that calculates the scalar value that is fed into the activation function $f(\cdot)$ (c.f. \eqref{McCulloch_Pitts_agg},  \eqref{nn}).\\
    $\mathcal{L}(\cdot)$ & A loss function to calculate the quality of the model output, typically comparing a prediction $\boldsymbol{\hat{y}}$ with a teaching signal $\boldsymbol{y}$, i.e. $\mathcal{L}(\boldsymbol{\hat{y}}, \boldsymbol{y})$\\
\end{tabular}
\renewcommand{\arraystretch}{1}






