Deep Learning Systeme haben in den letzten Jahren beeindruckene Resultate erzielt und können unter anderem mit hoher Qualität Texte übersetzen, Unterhaltungen führen oder Bilder generieren. Diese Systeme werden typischerweise mit Backpropagation of Error über sämtliche Netzwerklayer hinweg als ein grosses System trainiert. Dieser Prozess ist zwar für spezifische Task wie Klassifizierung sehr effizient, ist aber neurowissenschaftlich nicht plausibel und hat verschiedenste Schwächen wie fehlende Robustheit sowie Interpretierbarkeit und nicht die Fähigkeit kausale Schlussfolgerungen zu ziehen.

In dieser Thesis werden neurowissenschaftliche Erkenntnisse identifiziert, die für die Intelligenz von Lebewesen wie dem Mensch elementar sind und auf Deep Learning Systeme übertragen. Der Fokus liegt dabei auf dem visuellen Cortex als biologische Inspirationsquelle und Deep Learning Architekturen zur Bildverarbeitung als Zielsystem. Neurowissenschaftliche Erkenntnisse beziehen sich auf biologische Neuronen, welche sich gegenseitig durch zeitabhängige Spannungsspitzen gegenseitig anregen oder hemmen. Deep Learning Systeme hingegen haben keine Zeitdynamik und haben Aktivierungen basierend auf Fliesskommazahlen anstelle von Stromimpulsen. Folglich besteht ein grosser Beitrag dieser Arbeit darin die neurowissenschaftlichen Erkenntnisse in den Kontext von Deep Learning zu übertragen.

Die abgeleiteten Erkenntnisse werden in Form von zwei konkreten Modellen implementiert. Dabei besteht darin, neue Architekturideen zu demonstrieren und nicht irgendwelche Metriken von Benchmark-Datensätzen zu überbieten. Zudem orientieren sich die vorgeschlagenen Architekturen nahe am bewährten Deep Learning Framework und sind nicht als biologisch plausible Systeme zu verstehen. Konkret wird ein Modell mit vertikaler und ein Modell mit horizontaler Selbst-Organisation vorgeschlagen. Das Modell mit vertikaler Selbst-Organisation optimiert jedes Modell-Layer separat. Dabei werden neue Konzepte vorgestellt, die es erlauben alle Layer gleichzeitig zu trainieren und trotzdem hierarchische Features über die Layer hinweg zu erlernen. Das zweite Modell mit horizontaler Selbst-Organisation teilt die Eingabedaten in kleinere Einheiten auf und verteilt diese auf unabhängig trainierte Modelle. Dabei sieht jedes Modell nur ein Teil der Eingabedaten und ist auf eine lokale Interaktion mit seinen Nachbarmodellen angewiesen, um eine passende Bildrepräsentation abzuleiten.

TODO: Schlussfolgerung / Erkenntnisse sobald Experimente abgeschlossen sind.