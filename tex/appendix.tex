\chapter{Image Style Transfer}\chlbl{image_style_transfer}
In the field of image style transfer\sidecite{Gatys_Ecker_Bethge_2015}, such correlations are used to generate images.
An image is created based on two images, one image providing the content (i.e. the ``content-image'') and the other image providing the style (i.e. the ``style-image'').
To generate the content of the image, a random image can be fed into the model and a simple distance-based loss function such as the Euclidean norm can be used to calculate the difference between the model output and the content-image.
Over time, the model would learn to output the content-image.
However, only the content from the content-image and not its style is needed.
A solution to this problem is using Convolutional feature maps as they capture spatial information of an image without containing the style information.
Therefore, the distance loss is not calculated between the content-image and the model output but between the feature maps of the content-image and the output image.
The second task is to transfer the style from the style-image to the model output.
The style of an image can be described by the means and correlations across the different feature maps.
This information about the image style can be calculated with a Gram matrix.
A Gram matrix is the dot product\sidenote{the dot product between two vectors can be seen as similarity metric; it gets bigger if two vectors are more similar} of the flattened feature vectors from a convolutional feature map (i.e. the dot product between all channels within a feature map).
For example, a convolutional layer could have multiple channels. The first channel could have a high activation for black and white stripes in horizontal direction while the second channel could have a high activation for black and white stripes in vertical direction. If both channels activate together for the same input and thus have a high correlation, there is a high possibility that the image might contain the style of a chessboard (i.e. black and white checkered). A third channel, for example, could have a high activation for eyes. If this channel has a low correlation with the first channel but a high correlation with the second channel (i.e. black and white stripes in vertical direction), the input image might contain the face of a zebra and thus capture a ``zebra-style''.
Similar to the content-transfer, a distance loss such as the Euclidean distance can be used to compare the Gram matrices of the style-image and the output-image.
Thus, the content is transferred by comparing entire feature maps and the style by comparing the correlations between the channels per feature map (i.e. by comparing gram matrices).